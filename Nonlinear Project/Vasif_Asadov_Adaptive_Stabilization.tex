% Options for packages loaded elsewhere
% Options for packages loaded elsewhere
\PassOptionsToPackage{unicode}{hyperref}
\PassOptionsToPackage{hyphens}{url}
\PassOptionsToPackage{dvipsnames,svgnames,x11names}{xcolor}
%
\documentclass[
  letterpaper,
  DIV=11,
  numbers=noendperiod]{scrartcl}
\usepackage{xcolor}
\usepackage{amsmath,amssymb}
\setcounter{secnumdepth}{-\maxdimen} % remove section numbering
\usepackage{iftex}
\ifPDFTeX
  \usepackage[T1]{fontenc}
  \usepackage[utf8]{inputenc}
  \usepackage{textcomp} % provide euro and other symbols
\else % if luatex or xetex
  \usepackage{unicode-math} % this also loads fontspec
  \defaultfontfeatures{Scale=MatchLowercase}
  \defaultfontfeatures[\rmfamily]{Ligatures=TeX,Scale=1}
\fi
\usepackage{lmodern}
\ifPDFTeX\else
  % xetex/luatex font selection
\fi
% Use upquote if available, for straight quotes in verbatim environments
\IfFileExists{upquote.sty}{\usepackage{upquote}}{}
\IfFileExists{microtype.sty}{% use microtype if available
  \usepackage[]{microtype}
  \UseMicrotypeSet[protrusion]{basicmath} % disable protrusion for tt fonts
}{}
\makeatletter
\@ifundefined{KOMAClassName}{% if non-KOMA class
  \IfFileExists{parskip.sty}{%
    \usepackage{parskip}
  }{% else
    \setlength{\parindent}{0pt}
    \setlength{\parskip}{6pt plus 2pt minus 1pt}}
}{% if KOMA class
  \KOMAoptions{parskip=half}}
\makeatother
% Make \paragraph and \subparagraph free-standing
\makeatletter
\ifx\paragraph\undefined\else
  \let\oldparagraph\paragraph
  \renewcommand{\paragraph}{
    \@ifstar
      \xxxParagraphStar
      \xxxParagraphNoStar
  }
  \newcommand{\xxxParagraphStar}[1]{\oldparagraph*{#1}\mbox{}}
  \newcommand{\xxxParagraphNoStar}[1]{\oldparagraph{#1}\mbox{}}
\fi
\ifx\subparagraph\undefined\else
  \let\oldsubparagraph\subparagraph
  \renewcommand{\subparagraph}{
    \@ifstar
      \xxxSubParagraphStar
      \xxxSubParagraphNoStar
  }
  \newcommand{\xxxSubParagraphStar}[1]{\oldsubparagraph*{#1}\mbox{}}
  \newcommand{\xxxSubParagraphNoStar}[1]{\oldsubparagraph{#1}\mbox{}}
\fi
\makeatother


\usepackage{longtable,booktabs,array}
\usepackage{calc} % for calculating minipage widths
% Correct order of tables after \paragraph or \subparagraph
\usepackage{etoolbox}
\makeatletter
\patchcmd\longtable{\par}{\if@noskipsec\mbox{}\fi\par}{}{}
\makeatother
% Allow footnotes in longtable head/foot
\IfFileExists{footnotehyper.sty}{\usepackage{footnotehyper}}{\usepackage{footnote}}
\makesavenoteenv{longtable}
\usepackage{graphicx}
\makeatletter
\newsavebox\pandoc@box
\newcommand*\pandocbounded[1]{% scales image to fit in text height/width
  \sbox\pandoc@box{#1}%
  \Gscale@div\@tempa{\textheight}{\dimexpr\ht\pandoc@box+\dp\pandoc@box\relax}%
  \Gscale@div\@tempb{\linewidth}{\wd\pandoc@box}%
  \ifdim\@tempb\p@<\@tempa\p@\let\@tempa\@tempb\fi% select the smaller of both
  \ifdim\@tempa\p@<\p@\scalebox{\@tempa}{\usebox\pandoc@box}%
  \else\usebox{\pandoc@box}%
  \fi%
}
% Set default figure placement to htbp
\def\fps@figure{htbp}
\makeatother





\setlength{\emergencystretch}{3em} % prevent overfull lines

\providecommand{\tightlist}{%
  \setlength{\itemsep}{0pt}\setlength{\parskip}{0pt}}



 


\KOMAoption{captions}{tableheading}
\makeatletter
\@ifpackageloaded{caption}{}{\usepackage{caption}}
\AtBeginDocument{%
\ifdefined\contentsname
  \renewcommand*\contentsname{Table of contents}
\else
  \newcommand\contentsname{Table of contents}
\fi
\ifdefined\listfigurename
  \renewcommand*\listfigurename{List of Figures}
\else
  \newcommand\listfigurename{List of Figures}
\fi
\ifdefined\listtablename
  \renewcommand*\listtablename{List of Tables}
\else
  \newcommand\listtablename{List of Tables}
\fi
\ifdefined\figurename
  \renewcommand*\figurename{Figure}
\else
  \newcommand\figurename{Figure}
\fi
\ifdefined\tablename
  \renewcommand*\tablename{Table}
\else
  \newcommand\tablename{Table}
\fi
}
\@ifpackageloaded{float}{}{\usepackage{float}}
\floatstyle{ruled}
\@ifundefined{c@chapter}{\newfloat{codelisting}{h}{lop}}{\newfloat{codelisting}{h}{lop}[chapter]}
\floatname{codelisting}{Listing}
\newcommand*\listoflistings{\listof{codelisting}{List of Listings}}
\makeatother
\makeatletter
\makeatother
\makeatletter
\@ifpackageloaded{caption}{}{\usepackage{caption}}
\@ifpackageloaded{subcaption}{}{\usepackage{subcaption}}
\makeatother
\usepackage{bookmark}
\IfFileExists{xurl.sty}{\usepackage{xurl}}{} % add URL line breaks if available
\urlstyle{same}
\hypersetup{
  pdftitle={Adaptive Stabilization},
  pdfauthor={Vasif Asadov},
  colorlinks=true,
  linkcolor={blue},
  filecolor={Maroon},
  citecolor={Blue},
  urlcolor={Blue},
  pdfcreator={LaTeX via pandoc}}


\title{Adaptive Stabilization}
\author{Vasif Asadov}
\date{}
\begin{document}
\maketitle


\section{5.1. Giriş. Basit Örnek}\label{giriux15f.-basit-uxf6rnek}

Adaptif kontrol problemini bu bölümde tanıtmak için, önce aşağıdaki
formdaki basit bir doğrusal olmayan sistemi ele alıyoruz:

\[
\dot{x} = f^{T}(x,t)\mu + u \tag{5.1}
\]

Burada \(x\in \mathbb{R}\) state; \(u \in \mathbb{R}\) input;
\(f : \mathbb{R}\times [t_0,\infty] \mapsto \mathbb{R}^{l}\) \(x\)'e
göre yerel olarak Lipschitz koşulunu sağlayan ve \(t\)'de uniform olan
bilinen bir fonksiyondur ve \(\mu \in \mathbb{R}^{l}\) bilinmeyen sabit
bir parametre vektörüdür. \(\mu\)'nun her bir bileşeninin herhangi bir
reel değer almasına izin veriyoruz. Dolayısıyla, eğer \[
g(x,d(t)) = f^{T}(x,t)\mu \tag{5.2}
\] olarak tanımlarsak ve \(d(t) = (t,\mu)\) alınır, ve \(d\) sınırsızdır
(unbounded) ve Bölüm 4'teki yaklaşım çalışmaz.

(5.2) biçimindeki belirsiz bir fonksiyonun, doğrusal parametreleştirme
özelliğini (linear parameterization property) sağladığı söylenir; çünkü
herhangi bir reel sayı \(\gamma\) için
\(f^{T}(x,t)(\gamma \mu) = \gamma \big(f^{T}(x,t)\mu\big)\) eşitliği
geçerlidir. Buna karşılık, \(x\in \mathbb{R}\) ve \(\mu\) sabit bir reel
sayı olmak üzere, belirsiz fonksiyon \(\sin(\mu x)\) doğrusal
parametreleştirme özelliğini sağlamaz.

(5.1) numaralı sistemde, eğer μ biliniyor olsaydı, aşağıdaki kontrol
yasası \[
u = - f^T(x,t) \mu - \rho x, \quad \rho>0 \tag{5.3}
\] kapalı çevrim sistemi \(\dot{x} = -\rho x\) şeklinde verir; bu da
asimptotik olarak kararlı doğrusal bir sistemdir. Ancak \(\mu\)
bilinmediğinden, kontrol yasası (5.3) uygulanamaz. Bunun yerine
aşağıdaki kontrol yasasını ele alacağız: \[
u = - f^{T}(x,t)\hat{\mu} - \rho x, \quad \rho > 0 \tag{5.4}
\] Burada \(\hat{\mu} \in \mathbb{R}^{l}\) \(\mu\)'nun bir tahmini
(estimation) olarak görülen sabit bir vektördür. Bu kontrol yasası
altında, kapalı çevrim sistem \[
\dot{x} = - f^{T}(x,t)\tilde{\mu} - \rho x \tag{5.5}
\] şeklindedir; burada \(\tilde{\mu} = \hat{\mu} - \mu\) parametre
tahmin hatası olarak adlandırılır. Şimdi, \(l=1\) ve \(f(x,t) = x\) olan
(5.5)'in en basit durumunu ele alalım. Bu durumda, (5.5) aşağıdaki
doğrusal sisteme indirgenir.

\[
\dot{x} = -(\tilde{\mu} + \rho)x. \tag{5.6}
\] Şimdi, \(\mu\) keyfi olarak büyük değerler alabileceğinden, \(\rho\)
ne kadar büyük seçilirse seçilsin, (5.6)'nın kararlılığı garanti
edilemez; çünkü \(\tilde{\mu} + \rho > 0\) koşulu garanti edilemez.
Dolayısıyla, (5.4) biçimindeki statik durum geri beslemeli kontrol
yasası bu durum için çalışmaz. Bunun ardından, aşağıdaki dinamik durum
geri beslemeli kontrol yasasına geçiyoruz:

\[
u = - f^{T}(x,t)\hat{\mu} - \rho x, \quad \rho > 0
\]

\[
\dot{\hat{\mu}} = \Lambda x f(x,t)
\tag{5.7}
\]

Burada \(\hat{\mu} \in \mathbb{R}^l\) sabit bir vektör değil, (5.7)'nin
ikinci denklemi tarafından yönetilen zamana bağlı bir fonksiyondur ve
\(\Lambda \in \mathbb{R}^{l \times l}\) simetrik ve pozitif tanımlı
(positive definite) sabit bir matristir. \(\hat{\mu}, \mu\)'nun dinamik
tahmini (dynamic estimation) veya kısaca tahmin olarak adlandırılır;
(5.7)'nin ikinci denklemi ise parametre güncelleme yasası (parameter
update law) ya da kısaca güncelleme yasası olarak adlandırılır. (5.1)
numaralı sistem ile (5.7) numaralı kontrol yasasının birleşimi, kapalı
çevrim (closed-loop) sistem olarak adlandırılır ve aşağıdaki şekli alır:

\[
\begin{aligned}
\dot{x} &= - f^{T}(x,t)\hat{\mu} - \rho x \\
\dot{\hat{\mu}} &= \Lambda x f(x,t)
\end{aligned}
\tag{5.8}
\]

Şimdi, kapalı çevrim sistemin şu özelliğe sahip olduğunu göstereceğiz:
herhangi bir başlangıç koşulu \(x(0)\) ve \(\hat{\mu}(0)\) için, kapalı
çevrim sistemin (5.8) çözümü sınırlıdır ve
\(\lim_{t \to \infty} x(t) = 0\) olur. Bu amaçla, kapalı çevrim sistemi
(5.8) için aşağıdaki Lyapunov fonksiyonu adayını ele alıyoruz:

\[
W(x, \tilde{\mu}) = x^2/2 + \tilde{\mu}^T \Lambda^{-1} \tilde{\mu}/2 \tag{5.9}
\]

Kapalı çevrim sistemin durum yörüngesi boyunca \(W(x, \tilde{\mu})\)
fonksiyonunun türevi

\[
\dot{W}(x, \tilde{\mu}) = x\dot{x} + \tilde{\mu}^T \Lambda^{-1} \dot{\tilde{\mu}} \\
= -\rho x^2 - \tilde{\mu}^T x f(x,t) + \tilde{\mu}^T \Lambda^{-1} \dot{\hat{\mu}} \\
= -\rho x^2 \le 0
\tag{5.10}
\]

Dolayısıyla, durum değişkenleri \(x\) ve \(\hat{\mu}\) sınırlıdır.
Ayrıca, Teorem 2.5 uyarınca, \(\lim_{t \to \infty} x(t) = 0\) elde
edilir.

\subsection{Theorem 5.1}\label{theorem-5.1}

Sistem (5.1) için küresel adaptif kararlılık problemi, aşağıdaki
denetleyici tarafından çözülmektedir:

\[
u = -f^T(x,t)\hat{\mu} - \rho x, \quad \rho > 0
\]

\[
\dot{\hat{\mu}} = \Lambda x f(x,t), \quad \Lambda = \Lambda^T > 0
\]

\section{5.2. Adaptive stabilization. Tuning
functions.}\label{adaptive-stabilization.-tuning-functions.}

\textbf{Küresel Adaptif Regülasyon Problemi (GARP)} (2.7)-(2.9)
şeklindeki doğrusal olmayan bir kontrol sistemi verildiğinde, (1.11)
şeklinde bir denetliyici tasarla ki, kapalı çevrim sisteminin (2.10)
herhangi bir başlanğıç koşulu \(x_c(0)\) için, kapalı çevrim sisteminin
durum yörüngesi \(x_c(t)\) sınırlı kalsın ve
\(lim_{t \rightarrow \infty} y(t) = 0\) sağlansın. GARP tanımında,
yalnızca performans çıktısı \(y\)'nin asimptotik olarak orijine
yaklaşması şart koşulmaktadır. Eğer \(y=x\) ise, bu durumda GARP ayrıca
\textbf{Küresel Adaptif Kararlılık Problemi (GASP)} olarak adlandırılır.
Açıkça görülmektedir ki, sistem (5.1), \(y_m = y = x\) olmak üzere
(2.7)-(2.9) sistemlerinin özel bir durumudur.

\phantomsection\label{fig:example}
\includegraphics[width=0.6\linewidth,height=\textheight,keepaspectratio]{eqs.png}

Bu bölümde aşağıdaki lower triangular nonlinear sistemi ele alacağız:

\[\dot{x}_i = f_i^T(\vec{x_i}, t)\mu + x_{i+1}, \quad i=1,\dots, r \tag{5.13}\]
Burada \(\vec{x}_i = col(x_1,\dots, x_i)\) state vektoru
(\(x_i \in \mathbb{R}\)); \(u := x_{r+1} \in \mathbb{R}\) input ve
\(\mu \in \mathbb{R}^l\) bilinmeyen konstant parametre vektörüdür.
\(r=1\) olan fonksiyonu önceki bölümde inceledik. O örnekteki gibi,
bizim asıl amacımız öyle bir controller ve update law bulmak ki,
Lyapunov fonksiyonunun türevi negatif semi-definit olsun. Tasarım
prosedürü r adımdan oluşmaktadır. Her bir adımda \(i =1, \dots, r\) için
öncdeki adımlara dayanarak iki fonksiyon, \(s_i\) ve \(\tau_i\)
tasarlanır. Eğer sistemin relative derecesi \(i\)-ye eşit olsaydı, bu
iki fonksiyon sırasıyla denetim yasasını ve parametre güncelleme
yasasını oluştururdu. Aksi halde, süreç yinelemeli (recursive) olarak
\(i=r\) oluncaya kadar devam eder; böylece gerçek denetim yasası
(control law) \(x_{r+1} = S_r(.)\) ve gerçek parametre güncelleme yasası
\(\dot{\hat{\mu}} = \tau_r(.)\) elde edilir. Literatürde \(\tau_i\)
fonksiyonları ayar (tuning) fonksiyonları olarak adlandırıldığından, bu
tasarım yaklaşımı ayar fonksiyonları yaklaşımı (tuning functions
approach) olarak isimlendirilir ve \emph{Önerme 4.1}'in bir genellemesi
olarak değerlendirilebilir.

Daha detaylı araştırma için aşağıdaki dinamik koordinat dönüşümünü
tanımlayalım:

\[
\begin{aligned}
\chi_1 &= x_1 \\
\chi_{i+1} &= x_{i+1} - s_i(\vec{x}_i,\hat{\mu},t),
\qquad i = 1,\ldots,r \\
\dot{\hat{\mu}} &= \tau_r(\vec{x}_r,\hat{\mu},t)
\end{aligned}
\tag{5.14}
\] Burada \(s_i, i=1,\dots, r\) ve \(\tau_r\) yeterince düzgün
(sufficiently smooth) fonksiyonlardır. Eğer
\(\zeta_i = col(\vec{\chi}_i, \hat{\mu})\) ve
\(\vec{\chi}_i = col(\chi_1,..., \chi_i)\) olarak tanımlanırsa, (5.13)
ile verilen plant ve (5.14) ile verilen dönüşümden oluşan sistem
\(\dot{\zeta}_r = \phi_r(\zeta_r, \chi_{r+1},t)\) şeklinde ifade
edilebilir. Bu durumda, \emph{Proposition 4.1} aşağıdaki biçimde
genelleştirilebilir.

\textbf{Proposition 5.1} Eğer yeterince düzgün fonksiyonlar
\(s_i, i=1,\dots, r\) ve \(\tau_r\) mevcut olup, kapalı çevrim
sisteminin herhangi bir başlangıç koşulu \(\zeta_r(0)\) için,
\(\dot{\zeta} = \phi_r(\zeta_r, 0, t)\) sisteminin durumu \(\zeta_r\)
sınırlı (bounded) kalıyor ve
\(lim_{t \rightarrow \infty} {\vec{\chi_r}} = 0\) sağlanıyorsa, o hâlde
sistem (5.13) için \(y=x_1\) olmak üzere Küresel Adaptif Regülasyon
Problemi (GARP), aşağıdaki dinamik durum geri beslemeli denetleyici
(closed loop feedback controller) ile çözülmüş olur:

\[
u = S_r(\vec{x}_r, \hat{\mu}, t) \quad \newline
\dot{\hat{\mu}} = \tau_r(\vec{x_r}, \hat{\mu}, t) 
\tag{5.15}
\] Buna ek olarak, eğer
\(s_i(0,\hat{\mu}, t) = 0, \quad i=1,\dots, r-1\) koşulu sağlanıyorsa,
sistem (5.13) için \(y = \vec{\chi}_r\)olmak üzere Küresel Adaptif
Kararlılık Problemi (GASP) de aynı denetleyici tarafından çözülmüş olur.

Şimdi, \(i=1,\dots, r\) için \(s_i\) ve \(\tau_i\) fonksiyonlarını elde
etmek amacıyla özyinelemeli (recursive) bir prosedür tanıtıyoruz.
\(s_1\) ve \(\tau_1\)'in inşası, önceki bölümden esinlenmiştir; yani,

\[
s_1(x_1, \hat{\mu}, t) = -f_1^T(x_1, t)\hat{\mu} - \rho_1 x_1, \quad \rho_1 \ge 5/4 
\] \[
\tau_1(x_1, \hat{\mu}, t) = \Lambda x_1 f_1(x_1,t), \quad \Lambda = \Lambda^T > 0 
\tag{5.16}
\]

Kalan fonksiyonlar ise \(i=2,…,r\) için aşağıdaki şekilde rekursiv
olarak oluşturulur:

\[
\begin{align}
s_i(\vec{x}_i,\hat{\mu},t)
&= - q_i^{\top}(\vec{x}_i,\hat{\mu},t)\,\hat{\mu}
   - \rho_i x_i
   + v_i(\vec{x}_i,\hat{\mu},t),
\\
&\qquad \rho_i \ge \frac{9}{4}
\tag{5.17}
\end{align}
\]

\[
\begin{align}
\tau_i(\vec{x}_i,\hat{\mu},t)
&= \tau_{i-1}(\vec{x}_{i-1},\hat{\mu},t)
   + \Lambda\, \chi_i\, q_i(\vec{x}_i,\hat{\mu},t)
\tag{5.18}
\end{align}
\] Burada \(v_i\) ve \(\varrho_i\) aşağıdakı gibi tanımlanmıştır.

\[
\begin{align}
v_i(\vec{x}_i,\hat{\mu},t)
&= \sum_{j=1}^{i-1}
   \frac{\partial s_{i-1}(\vec{x}_{i-1},\hat{\mu},t)}{\partial x_j}
   \, x_{j+1}
 + \frac{\partial s_{i-1}(\vec{x}_{i-1},\hat{\mu},t)}{\partial t}
\\[6pt]
&\quad
 + \frac{\partial s_{i-1}(\vec{x}_{i-1},\hat{\mu},t)}{\partial \hat{\mu}}
   \, \tau_i(\vec{x}_i,\hat{\mu},t)
\\[6pt]
&\quad
 + \sum_{j=1}^{i-2}
   x_{j+1}
   \frac{\partial s_j(\vec{x}_j,\hat{\mu},t)}{\partial \hat{\mu}}
   \, \Lambda \, \varrho_i(\vec{x}_i,\hat{\mu},t)
\tag{5.19}
\end{align}
\]

\[
\begin{align}
\varrho_i(\vec{x}_i,\hat{\mu},t)
&= f_i(\vec{x}_i,t)
   - \sum_{j=1}^{i-1}
     \frac{\partial s_{i-1}(\vec{x}_{i-1},\hat{\mu},t)}{\partial x_j}
     \, f_j(\vec{x}_j,t)
\tag{5.20}
\end{align}
\]

Şimdi ise bu bölümün esas sonucunu aşağıdaki teorem ile özetleyebiliriz.

\subsection{Theorem 5.2.}\label{theorem-5.2.}

Performans çıktısı \(y=x_1\) olan sistem (5.13) için Küresel Adaptif
Regülasyon Problemi (GARP), (5.16)--(5.18)'de tanımlanan ve ayrıca
Algoritma 5.1'de özetlenen \(s_r\) ve \(\tau_r\) fonksiyonları
kullanılarak, (5.15) ile verilen denetleyici tarafından çözülmektedir.
Buna ek olarak, eğer \(f_i(0,t) = 0, \quad 1 = 1, \dots, r-1\) koşulu
sağlanıyorsa, performans çıktısı \(y = \vec{\chi}_r\) olan sistem (5.13)
için Küresel Adaptif Kararlılık Problemi (GASP) de aynı denetleyici
tarafından çözülmektedir.

\subsection{Algorithm 5.1.}\label{algorithm-5.1.}

\textbf{Input:} \(f_1, i=1,\dots, r\).

\textbf{Output:} \(s_r \text{ ve } \tau_r\).

\textbf{Step 1:} \(i = 1\) için \(s_1, \tau_1\) bul.

\textbf{Step 2:} \(i = r\) ise Step 5'e git, aksi takdirde Step 3'e git.

\textbf{Step 3:} \(i = i + 1\) için sırayla
\(\varrho_i, \tau_i, v_i \text{ ve } s_i\) değerlerini bul.

\textbf{Step 4:} Step 2'ye git.

\textbf{Step 5:} Son.

\textbf{Örnek 5.1} Aşağıdaki ikinci dereceden doğrusal olmayan sistemi
ele alalım:

\[
\dot{x_1} = x_1^2 \mu_1+x_2 
\] \[
\dot{x_2} = sin(x_1 x_2)\mu_2 + u \\
\tag{5.24}
\] burada \(\mu = [\mu_1 \quad \mu_2]^T\) konstant parametre vektorudur.
Amaç, performans çıktısı \(y = [x_1, x_2]^T\) olan Küresel Adaptif
Kararlılık Problemi (GASP)'ni çözen denetleyici \(u\)'yu bulmaktır.

Denetleyiciyi açıkça inşa etmek için Algoritma 5.1-i izlememiz lazım.
İlk adım, (5.16)'ya dayanarak aşağıdaki fonksiyonları seçmektir:

\[
s_1(x_1, \hat{\mu}) = -x_1^2 \hat{\mu}_1 - 2x_1 \\
\tau_1(x_1, \hat{\mu}) = [x_1^2 \quad 0] ^T
\]

Bir sonraki adımda, (5.17)--(5.20) numaralı bağıntılar kullanılarak
sırasıyla \(q_2, \tau_2, v_2\) ve \(s_2\) hesaplanacaktır.

\[
\begin{align}
\varrho_2(\vec{x}_2,\hat{\mu})
= f_2(\vec{x}_2)
   - \frac{\partial s_1(\vec{x}_1,\hat{\mu})}{\partial x_1}\,
     f_1(\vec{x}_1)
\\
= 
\begin{bmatrix}
(2 x_1 \hat{\mu}_1 + 2)\, x_1^2 &
\sin(x_1 x_2)
\end{bmatrix}^{\top}
\end{align}
\]

daha sonra

\[
\begin{align}
\tau_2(\vec{x}_2,\hat{\mu})
&= \tau_1(x_1,\hat{\mu})
   + \bigl[x_2 - s_1(x_1,\hat{\mu})\bigr]\,
     \varrho_2(\vec{x}_2,\hat{\mu})
\\[8pt]
&=
\begin{bmatrix}
x_1^3
+ (2 x_1 \hat{\mu}_1 + 2)\, x_1^2
  \bigl(x_2 + x_1^2 \hat{\mu}_1 + 2 x_1\bigr)
\\[6pt]
\sin(x_1 x_2)\,
\bigl(x_2 + x_1^2 \hat{\mu}_1 + 2 x_1\bigr)
\end{bmatrix}.
\end{align}
\]

Elde ettiğimiz fonksiyonlarla \(v_2\) ve \(s_2\)-yi de bulalım.

\[
\begin{align}
v_2(\vec{x}_2,\hat{\mu})
&=
\frac{\partial s_1(\vec{x}_1,\hat{\mu})}{\partial x_1}\, x_2
+ \frac{\partial s_1(\vec{x}_1,\hat{\mu})}{\partial \hat{\mu}}\,
  \tau_2(\vec{x}_2,\hat{\mu})
\\[8pt]
&=
- (2 x_1 \hat{\mu}_1 + 2)\, x_2
-
\begin{bmatrix}
x_1^2 & 0
\end{bmatrix}
\begin{bmatrix}
x_1^3
+ (2 x_1 \hat{\mu}_1 + 2)\, x_1^2
  (x_2 + x_1^2 \hat{\mu}_1 + 2 x_1)
\\[6pt]
\sin(x_1 x_2)\,
(x_2 + x_1^2 \hat{\mu}_1 + 2 x_1)
\end{bmatrix}
\\[10pt]
&=
- (2 x_1 \hat{\mu}_1 + 2)\, x_2
- x_1^4
\Bigl(
x_1
+ (2 x_1 \hat{\mu}_1 + 2)
  (x_2 + x_1^2 \hat{\mu}_1 + 2 x_1)
\Bigr).
\end{align}
\] ve

\[
\begin{align}
s_2(\vec{x}_2,\hat{\mu})
&=
- \varrho_2^{\top}(\vec{x}_2,\hat{\mu})\,\hat{\mu}
- \rho_2 \bigl[x_2 - s_1(x_1,\hat{\mu})\bigr]
+ v_2(\vec{x}_2,\hat{\mu})
\\[10pt]
&=
-
\begin{bmatrix}
(2 x_1 \hat{\mu}_1 + 2)\, x_1^2 &
\sin(x_1 x_2)
\end{bmatrix}
\hat{\mu}
- \frac{9}{4}\,\bigl(x_2 + x_1^2 \hat{\mu}_1 + 2 x_1\bigr)
\\[6pt]
&\quad
- (2 x_1 \hat{\mu}_1 + 2)\, x_2
- x_1^4
\Bigl(
x_1
+ (2 x_1 \hat{\mu}_1 + 2)
  \bigl(x_2 + x_1^2 \hat{\mu}_1 + 2 x_1\bigr)
\Bigr).
\end{align}
\]

Şimdi ise kontrolcüyü tasarlayabiliriz:

\[
u = s_2(\vec{x_2}, \hat{\mu})\\
\dot{\hat{\mu}} = \tau_2(\vec{x}_2, \hat{\mu})
\]

Denetleyicinin performansı \(\mu = [2,4]^T\) için aşağıda
gösterilmiştir. Başlangıç durum değerleri \(x(0) = [-2, 18]^T\) ve
\(\hat{\mu}(0) = [0,10]^T\) olarak seçilmiştir. Kapalı çevrim sisteminin
durum vektörü x'in, asimptotik olarak orijindeki denge noktasına
yakınsadığı ve tahmin edilen parametre vektörü \(\hat{\mu}\)'nun sınırlı
kaldığı, ancak gerçek parametre değeri \(\mu\)'ye yakınsamadığı
gözlemlenmektedir.

\pandocbounded{\includegraphics[keepaspectratio]{images/clipboard-3414064342.png}}

\section{5.3. Robust Adaptive
Stabilization}\label{robust-adaptive-stabilization}

Bu bölümde, hem bozucular hem de bilinmeyen parametreler içeren belirsiz
doğrusal olmayan sistemlerin bir sınıfını ele almak için robust ve
adaptif teknikleri birleştiriyoruz. Sistemler aşağıdaki gibi
tanımlanmaktadır:\\

\[
\begin{equation}\begin{aligned}\dot{z} &= q\bigl(z, x_1, d(t)\bigr) \\\dot{x}_1 &= f_1\bigl(z, x_1, d(t)\bigr) + b f_a^{T}(x_1,t)\mu + b x_2 \\\dot{x}_i &= f_i(\vec{x}_i) + x_{i+1}, \quad i = 2, \ldots, r\end{aligned}\tag{5.25}\end{equation}
\]

Burada \(z \in \mathbb{R}^n\) ve \(\vec{x}_i = col(x_1, ..., x_i)\) olup
\(x_i \in \mathbb{R}, i = 1,...,r\) olmak üzere durum değişkenleridir.
\(u := x_{r+1}\) girdidir. \(\mu \in \mathbb{R}^{l_1}\) bilinmeyen sabit
parametredir ve
\(d:[0, \infty] \rightarrow \mathbb{D} \subset \mathbb{R}^{l_2}\) olup,
burada \(\mathbb{D}\) kompakt bir kümedir ve bilinmeyen parametreleri
veya bozucuları temsil eder. Fonksiyonlar
\(q, f_a, f_i, i = 1, ... , r\) yeterince düzgündür (türevleri var) ve
aşağıdaki koşulları sağlar:

\[
\begin{equation}\begin{aligned}q(0,0,d) &= 0 \\f_1(0,0,d) &= 0, \quad \text{tüm } d \in \mathbb{D} \text{ için} \\f_i(0) &= 0, \quad i = 2, \ldots, r\end{aligned}\end{equation}
\]

Alt index \(a\), \(f_a^T(x_1, t)\mu\) teriminin adaptif kontrol ile ele
alınacağını ifade eder. (5.25)'te, \(q\) fonksiyonu tam olarak
bilinmeyebilir ve/veya \(z\) durumu geri besleme kontrolü için
kullanılmayabilir. Bu nedenle, \(z\)'yi yöneten dinamikler dinamik
belirsizlik olarak kabul edilir. \(\mu = 0\) olduğunda, sistem (5.25)
bir önceki bölümde incelenen filtre genişletilmiş forma (filter extended
form) indirgenir.

\textbf{Assumption 5.1}.\(\bar{b} \geq b \geq \underline{b}\) olacak
şekilde iki pozitif sabit \(\bar{b}\) ve \(\underline{b}\) vardır.

\subsection{5.3.1. Systems with Relative Degree
One}\label{systems-with-relative-degree-one}

Öncelikle, \textbf{bağıl derecesi} (relative degree) \(r=1\) olan (5.25)
numaralı sistemi ele alıyoruz; yani,

\[
\begin{equation}\begin{aligned}\dot{z} &= q\bigl(z, x, d(t)\bigr) \\\dot{x} &= f\bigl(z, x, d(t)\bigr) + b f_a^{T}(x,t)\mu + b u\end{aligned}\tag{5.26}\end{equation}
\] \(y=x\) durumunda ortaya çıkan non-trivial ters dinamikleri ele
alabilmek için, Varsayım 4.1 ile aynı rolü oynayan aşağıdaki varsayıma
ihtiyaç duyarız.

\textbf{Assumption 5.2}. Alt sistem \[
\dot{z}=q(z,x,d)
\] için bir İSS Lyapunov fonksiyonu \(V(z)\) vardır; yani

\[
\begin{equation}V(z) \sim \{ \underline{\alpha}, \bar{\alpha}, \alpha, \sigma \mid \dot{z} = q(z,x,d) \}\end{equation}
\]

ve

\[
\begin{equation}\limsup_{s \to 0^{+}} \frac{s^{2}}{\alpha(s)} < \infty,\qquad\limsup_{s \to 0^{+}} \frac{\sigma(s)}{s^{2}} < \infty\end{equation}
\]

İlk adım, Bölüm 4.2'deki prosedürü tekrarlamaktır. Özellikle,
Appendix'teki (11.13) denklemi kullanılarak, yeterince düzgün ve negatif
olmayan iki fonksiyon \(m_1\)\hspace{0pt} ve \(m_2\)\hspace{0pt}
bulunabilir; öyle ki

\[
\begin{equation}\lvert f(z,x,d) \rvert \le m_1(z)\lVert z \rVert + m_2(x)\lvert x \rvert,\qquad \forall d \in \mathbb{D}\tag{5.27}\end{equation}
\]

Yeterince düzgün (sufficiently smooth) bir fonksiyon seçelim; öyle ki

\[
\begin{equation}\Delta(z) \ge 1 + m_1^{2}(z)\tag{5.28}\end{equation}
\]

Corrolary 2.2-ye göre, bazı sınıf \(K_{\infty}\) fonksiyonlari
\(\alpha'\) ve \(\bar{\alpha}'\) için

\[
\alpha'(\lVert z \rVert) \le V'(z) \le \bar{\alpha}'(\lVert z \rVert)
\]

koşulunu sağlayan, sürekli türevlenebilir bir fonksiyon \(V'(z)\)
mevcutttur. Bu fonksiyon, \(\dot{z} = q(z,x,d)\) durum yörüngesi boyunca
aşağıdaki eşitsizliği sağlar:

\[
\dot{V}'(z) \le -\Delta(z)\lVert z \rVert^{2} + \chi(x)x^{2}\tag{5.29}
\]

Burada \(\chi\), yeterince düzgün bir fonksiyondur. Daha sonra,
aşağıdaki koşulu sağlayan yeterince düzgün bir fonksiyon \(\rho(x)\)
tanımlanır:

\[
\rho(x) \ge \bigl[\chi(x) + m_2(x) + 5/4\bigr]/b\tag{5.30}
\]

\subsubsection{Theorem 5.3}\label{theorem-5.3}

\(\mathbb{D}\)-nin önceden verilmiş (prescribed) kompakt bir küme olduğu
(5.26) numaralı sistemi ele alalım. Varsayım 5.1 ve Varsayım 5.2
altında, performans çıktısı \(y = \operatorname{col}(z, x)\) olan (5.26)
sisteminin GASP problemi, aşağıdaki kontrolcü ile çözülmektedir:

\[
u = - f_a^{T}(x, t)\hat{\mu} - \rho(x)x
\]

\[
\dot{\hat{\mu}} = \Lambda x f_a(x, t), \quad \Lambda = \Lambda^{T} > 0 \tag{5.31}
\]

Burada \(\rho\) fonksiyonu (5.30)'da verilmiştir.

\textbf{Ispat:}

Kapalı çevrim (closed-loop) sistem için aşağıdaki Lyapunov fonksiyonu
adayını tanımlayalım:

\[
W(z, x, \tilde{\mu}) = V'(z) + x^2/2 + b\,\tilde{\mu}^T \Lambda^{-1} \tilde{\mu}/2
\]

Burada \(\tilde{\mu} = \hat{\mu} - \mu\) parametre kestirim hatasıdır.
Doğrudan hesaplama ile, kapalı çevrim sistemin durum yörüngesi boyunca
\(W(z, x, \tilde{\mu})\) fonksiyonunun türevinin aşağıdaki
eşitsizlikleri sağladığı görülür:

\[
\begin{equation}\begin{aligned}\dot{W}(z,x,\tilde{\mu})&\le -\Delta(z)\lVert z \rVert^{2}     + \chi(x)x^{2}     + x\bigl[f(z,x,d) + b f_a^{T}(x,t)\mu + b u\bigr]     + b\tilde{\mu}^{T}\Lambda^{-1}\dot{\hat{\mu}} \\[0.6em]&\le -\Delta(z)\lVert z \rVert^{2}     + \chi(x)x^{2}     + x\bigl[f(z,x,d) - b\rho(x)x\bigr] \\[0.6em]&\quad     + bx\bigl[f_a^{T}(x,t)\mu - f_a^{T}(x,t)\hat{\mu}\bigr]     + b\tilde{\mu}^{T}\Lambda^{-1}\dot{\hat{\mu}} \\[0.6em]&\le -\Delta(z)\lVert z \rVert^{2}     + m_1^{2}(z)\lVert z \rVert^{2}     + x^{2}\bigl[\chi(x) + 1/4 + m_2(x) - b\rho(x)\bigr] \\[0.6em]&\quad     + b\tilde{\mu}^{T}       \bigl(-x f_a(x,t) + \Lambda^{-1}\dot{\hat{\mu}}\bigr) \\[0.6em]&\le -\lVert z \rVert^{2} - x^{2}.\end{aligned}\tag{5.32}\end{equation}
\]

Artık kapalı çevrim (closed-loop) sistemin kararlılığı, \textbf{Teorem
2.5} kullanılarak tesis edilebilir. Özellikle,
\(\lim_{t \to \infty} \operatorname{col}(z(t), x(t)) = 0\)olduğu elde
edilir. Böylece GASP problemi çözülmüş olur. Son olarak, parametre
\(\hat{\mu}\)'nun yakınsama (convergence) problemi, aşağıdaki sonuç
(corollary) ile gösterildiği şekilde ele alınacaktır.

\subsubsection{Corollary 5.1.}\label{corollary-5.1.}

Teorem 5.3'te, verilen (5.26) numaralı sistem ile (5.31) numaralı
kontrolörden oluşan kapalı çevrim sistem, herhangi bir başlangıç durumu
için aşağıdaki özelliklere sahiptir:

\[
\lim_{t \to \infty} \dot{\hat{\mu}}(t) = 0, \tag{5.33}
\]

ve eğer \(d(t),\dot{d}(t),f_a(x, t)\)
\(\frac{\partial f_a(x, t)}{\partial t}\) ve
\(\frac{\partial f_a(x, t)}{\partial x}\) tüm \(t \ge 0\) için sınırlı
(bounded) ise, o zaman

\[
\lim_{t \to \infty} f_a^{T}(0, t)\,(\hat{\mu}(t) - \mu) = 0. \tag{5.34}
\]

Ayrıca, eğer \(f_a(0, t)\) kalıcı uyarım (persistently exciting,
\textbf{PE}) özelliğine sahipse, o halde

\[
\lim_{t \to \infty} \hat{\mu}(t) = \mu. \tag{5.35}
\]

\textbf{Ispat}

Teorem 5.3'te,
\(\lim_{t \to \infty} \operatorname{col}(z(t), x(t)) = 0\) sonucunun
ispatlandığı gösterilmiştir; bu da (5.33)'ü ima eder.
\(d(t), \dot{d(t)}, f_a(x,t)\),
\(\frac{\partial f_a(x, t)}{\partial t}\) ve
\(\frac{\partial f_a(x, t)}{\partial x}\) fonksiyonları sınırlı
olduğundan, \(\ddot{x}\) sınırlıdır ve dolayısıyla \(\dot{x}\) uniform
süreklidir. \textbf{Barbalat Lemması} ile,
\(\dot{x}(t) \to 0 \quad \text{iken } t \to \infty\) elde edilir; bu da
(5.34)'ü ima eder. Son olarak, \textbf{Lemma 2.4} ile, \(f_a(0, t)\)
kalıcı uyarım (PE) özelliğine sahipse (5.35) sağlanır

\subsection{5.3.2. Systems with High Relative
Degree}\label{systems-with-high-relative-degree}

Bu bölümde, \(r > 1\) olan genel (5.25) numaralı sistemi ele alıyoruz.
\(r > 1\) olduğunda, (5.25)'teki \(x_2\)\hspace{0pt} gerçek bir kontrol
girdisi değildir; ancak Teorem 5.3'te verilen (5.31) numaralı kontrolör
aşağıdaki koordinat dönüşümünü teşvik eder:

\[
\chi_2 = x_2 - s_1(x_1, \hat{\mu}, t), \quad s_1(x_1, \hat{\mu}, t) = - f_a^{T}(x_1, t)\hat{\mu} - \rho_1(x_1)x_1
\]

Burada \(\chi_2\) aşağıdaki denklemle yönetilmektedir:

\[
\begin{equation}\begin{aligned}\dot{x}_2&= f_2(\vec{x}_2) - \dot{s}_1(x_1,\hat{\mu},t) + x_3 \\[0.6em]&= f_2(\vec{x}_2)   - \frac{\partial s_1(x_1,\hat{\mu},t)}{\partial x_1}     \bigl[       f_1(z,x_1,d)       + b f_a^{T}(x_1,t)\mu       + b x_2     \bigr] \\[0.6em]&\quad   - \frac{\partial s_1(x_1,\hat{\mu},t)}{\partial \hat{\mu}}     \dot{\hat{\mu}}   - \frac{\partial s_1(x_1,\hat{\mu},t)}{\partial t}   + x_3 .\end{aligned}\tag{5.36}\end{equation}
\]

Burada \(\mu\)-ya ek olarak, parametre \(b\) de bilinmeyendir ve tahmin
edilmesi gerekmektedir. \(\hat{b}\), \(b\)'nin tahmini olarak
kabuledilirse ve \(\tilde{b} = \hat{b} - b\) ise tahmin hatası olur.

Tahmin \(\hat{b}\)'nin tanıtılmasıyla birlikte, denetim yasasını
tanımlayan çeşitli fonksiyonlar da \(\hat{b}\)'ye bağlı olabilir. Bu
nedenle, aşağıdaki dinamik koordinat dönüşümü tanımlanır:

\[
\begin{align}
\chi_1
&= x_1,
\\[6pt]
\chi_{i+1}
&= x_{i+1} - s_i(\vec{x}_i,\hat{\mu},\hat{b},t),
\qquad i = 1,\dots,r,
\\[8pt]
\dot{\hat{\mu}}
&= \tau_r(\vec{x}_r,\hat{\mu},\hat{b},t),
\\[6pt]
\dot{\hat{b}}
&= \varpi_r(\vec{x}_r,\hat{\mu},\hat{b},t).
\tag{5.37}
\end{align}
\]

Burada, daha sonra belirtilecek olan yeterince düzgün fonksiyonlar
\(s_i, i=1,..,r, \tau_r \text{ ve } \omega_r\) bulunmaktadır.
\(\zeta_i = \operatorname{col}(\vec{\chi}_i, \hat{\mu}, \hat{b})\)
olarak tanımlansın. \(\zeta_r\)`yi yöneten sistem
\(\dot{\zeta}_r = \varphi_r(\zeta_r, \chi_{r+1}, t)\) şeklinde
gösterilsin.

\textbf{Proposition 5.2}\\
Eğer \(i = 1, \ldots, r\) için yeterince düzgün fonksiyonlar \(s_i\) ve
\(\tau_r\) mevcutsa; öyle ki
\(\dot{\zeta}_r = \varphi_r(\zeta_r, 0, t)\) sistemin durumu \(\zeta_r\)
sınırlı (bounded) olsun ve \(\lim_{t \to \infty } \vec{\chi}_r = 0\)
sağlansın, o hâlde (5.25) numaralı sistem için performans çıktısı
\(y = x_1\) olan \textbf{GARP} problemi, aşağıda verilen dinamik
kontrolör ile çözülmektedir.

\[
\begin{equation}\begin{aligned}u &= s_r(\vec{x}_r,\hat{\mu},\hat{b},t) \\\dot{\hat{\mu}} &= \tau_r(\vec{x}_r,\hat{\mu},\hat{b},t) \\\dot{\hat{b}} &= \varpi_r(\vec{x}_r,\hat{\mu},\hat{b},t)\end{aligned}\tag{5.38}\end{equation}
\]

Ayrıca, \(i = 1, \dots, r-1\) için \(S_i(0, \mu, \hat{b}, t) = 0\)
olmasi halinde (5.25) numaralı sistem için performans çıktısı
\(y = \vec{x}_r\) olan \textbf{GASP} problemi, aynı kontrolcü ile
çözülmektedir..

Bu problemde de \(S_r, \tau_r\) ve \(\omega_r\) denklemlerini rekursiv
şekilde bulmamız gerekiyor. \(r=1\) olduğunda ilk adım bundan önceki
örnekte verildiği gibi olacaktır.

Bölüm 5.2'de olduğu gibi, \(s_r\), \(\tau_r\)\hspace{0pt} ve
\(\varpi_r\) fonksiyonlarını özyinelemeli (recursive) olarak bulmamız
gerekmektedir. İlk adım, \(r = 1\) olan önceki durumdan esinlenmiştir.
Yine, Ek'teki (11.13) bağıntısına göre aşağıdakinin geçerli olduğunu not
ederiz:

\[
|f_1(z, x_1, d)| \le m_1(z)\|z\| + m_2(x_1)|x_1|, \quad \forall d \in \mathbb{D} \tag{5.39}
\]

Burada \(m_1\)\hspace{0pt} ve \(m_2\)\hspace{0pt}, yeterince düzgün
(sufficiently smooth) ve negatif olmayan bazı fonksiyonlardır. Şimdi

\[
\dot{z} = q(z, x_1, d)  \tag{5.40}
\]

kabuledelim.

Varsayım 5.2 altında, değişen besleme (changing supply) fonksiyonu
tekniği (Sonuç 2.2) kullanılarak, bazı sınıf
\(\mathcal{K}_\infty\)\hspace{0pt} fonksiyonları \(\alpha'\) ve
\(\bar{\alpha'}\) için

\[
\alpha'(\|z\|) \le V'(z) \le \bar{\alpha}'(\|z\|)
\]

koşulunu sağlayan sürekli türevlenebilir bir fonksiyon \(V'(z)\)
mevcuttur; öyle ki,

\[
\dot{z} = q(z, x_1, d)
\]

durum yörüngesi boyunca aşağıdaki eşitsizlik sağlanır:

\[
\dot{V}'(z) \le -\Delta(z)\|z\|^2 + \chi(x_1)x_1^2 \tag{5.41}
\]

Burada \(\chi\) yeterince düzgün ve negatif olmayan bir fonksiyondur.
Eğer,

\[
\rho_1(x_1) \ge \big[\chi(x_1) + m_2(x_1) + 5/4 + (r-1)m_2^2(x_1)\big]/\underline{b} + \bar{b}/4 \tag{5.42}
\]

kabuledersek, \(r=1\) icin aşağıdaki denklemleri yazabiliriz.

\[
\begin{equation}\begin{aligned}s_1(x_1,\hat{\mu},\hat{b},t)&= - f_a^{T}(x_1,t)\hat{\mu} - \rho_1(x_1)x_1 \\\tau_1(x_1,\hat{\mu},\hat{b},t)&= \Lambda_{\mu} x_1 f_a(x_1,t) \\\varpi_1(x_1,\hat{\mu},\hat{b},t)&= 0\end{aligned}\tag{5.43}\end{equation}
\]

\(r=2,\dots,r\) için gereken denklemleri ise aşağıdaki gibi yazabiliriz:

\[
\begin{equation}\begin{aligned}s_i(\vec{x}_i,\hat{\mu},\hat{b},t)&= q_i(\vec{x}_{i-1},\hat{\mu},\hat{b},t)   - \rho_i(\vec{x}_{i-1},\hat{\mu},\hat{b},t)\,\chi_i   + \nu_i(\vec{x}_i,\hat{\mu},\hat{b},t) \\[0.6em]\tau_i(\vec{x}_i,\hat{\mu},\hat{b},t)&= \tau_{i-1}(\vec{x}_{i-1},\hat{\mu},\hat{b},t)   - \Lambda_{\mu}\,\chi_i\,     \frac{\partial s_{i-1}(\vec{x}_{i-1},\hat{\mu},\hat{b},t)}{\partial x_1}\,     f_a(x_1,t) \\[0.6em]\varpi_i(\vec{x}_i,\hat{\mu},\hat{b},t)&= \varpi_{i-1}(\vec{x}_{i-1},\hat{\mu},\hat{b},t)   - \Lambda_b\,\chi_i\,     \frac{\partial s_{i-1}(\vec{x}_{i-1},\hat{\mu},\hat{b},t)}{\partial x_1}\,     \bigl[f_a^{T}(x_1,t)\hat{\mu} + x_2\bigr]\end{aligned}\tag{5.44}\end{equation}
\]

Burada

\[
\begin{equation}\begin{aligned}\varrho_i(\vec{x}_{i-1},\hat{\mu},\hat{b},t)&=\Biggl( - \sum_{j=1}^{i-2}   x_{j+1}\,   \frac{\partial s_j(\vec{x}_j,\hat{\mu},\hat{b},t)}{\partial \hat{\mu}}\,   \Lambda_{\mu}\Biggr)\frac{\partial s_{i-1}(\vec{x}_{i-1},\hat{\mu},\hat{b},t)}{\partial x_1}\, f_a(x_1,t) \\[0.8em]&\quad+\Biggl( \hat{b} - \sum_{j=1}^{i-2}   x_{j+1}\,   \frac{\partial s_j(\vec{x}_j,\hat{\mu},\hat{b},t)}{\partial \hat{b}}\,   \Lambda_{b}\Biggr)\frac{\partial s_{i-1}(\vec{x}_{i-1},\hat{\mu},\hat{b},t)}{\partial x_1}\bigl[  f_a^{T}(x_1,t)\hat{\mu} + x_2\bigr]\end{aligned}\end{equation}
\]

ve

\[
\begin{equation}\begin{aligned}\nu_i(\vec{x}_i,\hat{\mu},\hat{b},t)&= - f_i(\vec{x}_i)   + \sum_{j=2}^{i-1}     \frac{\partial s_{i-1}(\vec{x}_{i-1},\hat{\mu},\hat{b},t)}{\partial x_j}     \bigl[       f_j(\vec{x}_j) + x_{j+1}     \bigr] \\[0.8em]&\quad   + \frac{\partial s_{i-1}(\vec{x}_{i-1},\hat{\mu},\hat{b},t)}{\partial \hat{\mu}}\,     \tau_i(\vec{x}_i,\hat{\mu},\hat{b},t) \\[0.8em]&\quad   + \frac{\partial s_{i-1}(\vec{x}_{i-1},\hat{\mu},\hat{b},t)}{\partial \hat{b}}\,     \varpi_i(\vec{x}_i,\hat{\mu},\hat{b},t) \\[0.8em]&\quad   + \frac{\partial s_{i-1}(\vec{x}_{i-1},\hat{\mu},\hat{b},t)}{\partial t}.\end{aligned}\end{equation}
\]

Yukardaki denklemlerde \(\Lambda_\mu = \Lambda_\mu^{T} > 0\) ve scalar
\(\Lambda_b > 0\) arbitrarily seçilebilir. Ayrıca \(i \geq 2\) için
\(S_i(\vec{x}_i, \hat{\mu}, \hat{b}, t)\) fonksiyonunun, aşağıdaki
teoremin ispatını kolaylaştırmak amacıyla üç terime ayrıldığına dikkat
edilir.

\subsubsection{Theorem 5.4.}\label{theorem-5.4.}

\(\mathbb{D}\)`nin önceden verilmiş (prescribed) kompakt bir küme olduğu
(5.25) numaralı sistemi ele alalım. Varsayım 5.1-i ve Varsayım 5.2-i ele
alarak, \(y = \operatorname{col}(z, x_1)\) olan (5.25) sisteminin
\textbf{GARP} problemi (5.38) numaralı kontröler ile çözülmektedir;
burada \(s_r\) ve \(\tau_r\) fonksiyonları (5.43)-(5.44)'le tanımlanmış
olup ayrıca Algoritma 5.2'de özetlenmiştir. Ayrıca, eğer
\(f_a(0, t) = 0\) ise performans çıktısı
\(y = \operatorname{col}(z, \vec{x}_r)\) olan (5.25) sisteminin
\textbf{GASP} problemi de aynı kontrolcü ile çözülmektedir.

\subsubsection{Algorithm 5.2.}\label{algorithm-5.2.}

\textbf{Input 5.2:}
\(f_i, i=1,\dots, r, f_a, \bar{b}, \underline{b}, \underline{\alpha}, \bar{\alpha}, \alpha, \sigma,\mathbb{D}\)

\textbf{Output:} \(s_r, \tau_r\) ve \(\varpi_r\)

\textbf{Step 1:} Verilen \(f_1(z,x_1, d)\) için (5.39)-cu denklemleri
kullanarak \(m_1\) ve \(m_2\)-ni bul.

\textbf{Step 2:} (5.40) denkleminden \(\Delta\) - nı bul ve
\(\underline{\alpha}', \bar{\alpha}', \chi = \text{algorithm 2.3} (\underline{\alpha}, \bar{\alpha}, \alpha, \sigma, \Delta)\).

\textbf{Step 3:} \(i=1\) olduğunda (5.43)'ü kullanarak \(s_1\) ve
\(\tau_1\)'i bul.

\textbf{Step 4:} \(i=r\) olduğunda, Step 7'e git, aksi takdirde Step 5'e
git.

\textbf{Step 5:} \(i=i+1\) olduğunda, (5.44)'cü denklemlerden
\(s_i, \tau_i\) ve \(\varpi_i\) değerlerini bul.

\textbf{Step 6:} Step 4'e git.

\textbf{Step 7:} Son.




\end{document}
